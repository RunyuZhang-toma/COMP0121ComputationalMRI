\documentclass[a4paper,11pt]{article}

\usepackage[margin=1in]{geometry} %choose a margin
\usepackage{graphicx} %insert pictures
\pdfoptionpdfminorversion=7 %stops graphicsx complaining
\usepackage{siunitx} %typeset SI units
\usepackage{units} %nicefrac
\usepackage{float} %puts pictures where I said
\floatplacement{figure}{H} % to place figure at the end
\usepackage{mathtools} %extra Maths commands
\usepackage{amssymb} %extra symbols
\usepackage{amsmath} %extra Maths commands
%\usepackage[compact]{titlesec} %reduces whitespace around section headings
\numberwithin{equation}{section} %numbers equations by sections, part of amsmath
% \numberwithin{figure}{section} %numbers figures by sections, part of amsmath
\usepackage{textcomp,gensymb} %extra symbols, textcomp removes warnings about \perthousand and \micro
\usepackage{bm} %bold in mathmode (text style bolding)
\usepackage{parskip} %no indents on new paragraphs
\setlength\parskip{\baselineskip} %tidies paragraphs
\usepackage{caption} %2 pics next to each other
\usepackage{subcaption}  %2 pics next to each other
\usepackage{color} %used in listings
\usepackage{listings} %formats code nicely

\newcommand{\diff}[2]{\frac{d #1}{d #2}}

\newcommand{\image}[4][10]{\begin{figure}[ht]
\centering
\includegraphics[width=#1cm]{./figures/#2.#3}
\caption{#4}
\label{fig:#2}
\end{figure}}

\begin{document}
\author{AUTHOR}
\title{Computational MRI (COMP0121) Coursework 1}
\date{\today}
\maketitle

\section{Spin excess}
\subsection{}
When a group of particle is placed in a magnetic field, the spin of each particle aligns parallel or anti-parallel to the external field.
The Spin excess is given by Boltzmann distribution
\begin{equation}
    \frac{N^-}{N^+} = e^{-E/kT}
    \label{equation:Boltzmann}
\end{equation}
where $N^+$ and $N^-$ are the number of spins in lower energy alignment and higher energy alignment, respectively. 
$E$ is the energy difference between the two spin energy levels, $k$, the Boltzmann constant, and T, the the absolute temperature.
The split of spin energy level under external magnetic field is determined by Larmor frequency $\omega$.
\begin{equation}
    E = h \nu
    \label{equation:photon_energy}
\end{equation}
where $h$ is the Planck constant, and $\omega$ is given by
\begin{equation}
    \omega = \gamma B
    \label{equation:Larmor}
\end{equation}
where $\gamma$ is the gyromagnetic ratio, and B, the magnetic field strength.
Combining Eq.\ref{equation:Boltzmann} to \ref{equation:Larmor} yields
\begin{equation}
    \frac{N^-}{N^+} = e^{-h\gamma B/kT}
\end{equation}

\subsection{}
Code for the computation of spin excess can be found in spin\_excess.m file. 

\section{Forced precession with an on-resonance RF field}
\subsection{}
The translation matrix for rotating a 3-D vector about the $z$-axis by some angle $\theta$ is
\begin{equation}
    R_z(\theta) = 
    \begin{bmatrix}
        \cos \theta & -\sin \theta  & 0 \\
        \sin \theta & \cos \theta & 0 \\
        0 & 0 & 1 \\
    \end{bmatrix}
\end{equation}

\subsection{}
Similarly,
\begin{equation}
    R_x(\theta) = 
    \begin{bmatrix}
        1 & 0 & 0 \\
        0 & \cos \theta & -\sin \theta \\
        0 & \sin \theta & \cos \theta \\
    \end{bmatrix}
\end{equation}

\subsection{}
and
\begin{equation}
    R_y(\theta) = 
    \begin{bmatrix}
        \cos \theta & 0 & \sin \theta \\
        0 & 1 & 0 \\
        -\sin \theta & 0 & \cos \theta \\
    \end{bmatrix}
\end{equation}

\subsection{}
A video visualising the forced precession of magnetization $\overrightarrow{M}=\left[0\ 0\ 1\right]'$ in rotating frame is shown in the file video2.4.avi.
A figure of that can be found in Fig. \ref{fig:figure2.4}.
Find the code producing these in Problem2.m.
Note that video and script files are numbered after the corresponding problem number. 

\section{Free precession in the main static magnetic field}
\subsection{}
A video for the forced precession in laboratory frame can be find in video3.1.avi.
A figure of that can be found in Fig. \ref{fig:figure3.1}.

\subsection{}
The free precession with the effect of $T_1$ and $T_2$ relaxation is governed by the Bloch equation,

\begin{equation}
    \diff{\overrightarrow{M}}{t} = \gamma \overrightarrow{M} \times \overrightarrow{B}  - \frac{\overrightarrow{M}_\perp }{T_2} + \frac{(M_0-M_z)\hat{z}}{T_1}
    \label{equation:Bloch}
\end{equation}

This ordinary differential equation can be solve by Euler method directly.
To make use of the rotation matrix developed in the last problem, the first term of Eq. \ref{equation:Boltzmann} which merely results in rotation of $\overrightarrow{M}$ around $\overrightarrow{B}$ is solved analytically.
This ensures that the choice of step size in Euler method will not otherwise affect the precision of the rotation.
The reduced Bloch equation (in the on-resonance Rotating frame)
\begin{equation}
    \diff{\overrightarrow{M}^*}{t} = - \frac{\overrightarrow{M}_\perp^* }{T_2} + \frac{(M_0-M^*_z)\hat{z}}{T_1}
    \label{equation:reduced Bloch}
\end{equation}
is solved instead, which governs only relaxation process, where $\overrightarrow{M}^*$ is magnetization in the on-resonance Rotating frame.
The Euler's method is the following iteration,
\begin{equation}
    \overrightarrow{M}^*_{k+1}= \overrightarrow{M}^*_k + \Delta t \left(-\frac{\overrightarrow{M}_{\perp,k}^*}{T_2} + \frac{(M_0-M^*_{z,k})\hat{z}}{T_1}\right)
\end{equation}
The magnetization in the laboratory frame can be calculated by translation, in this case, rotation around $z$ axis.
\begin{equation}
    \overrightarrow{M}_{k} = R_z(\theta_k)\overrightarrow{M}^*_{k}
\end{equation}
where $\theta_k$ is the rotation angle with respect to the initial orientation of magnetization at step $k$. 
The parameters used in this report are (the same everywhere unless explicit specified): $T_1$ = 10 ms, $T_2$ = 5 ms, and $\omega_0$ = 4$\pi$ KHz. The results are in video3.2.avi and Fig. \ref{fig:figure3.2}.

\subsection{}

This is to solve Eq. \ref{equation:reduced Bloch}. Find the results in in video3.3.avi and Fig. \ref{fig:figure3.3}.

\subsection{}
Find the results in in video3.4.avi and Fig. \ref{fig:figure3.4}.

\subsection{}
Find the results in in video3.5.avi and Fig. \ref{fig:figure3.5}.

\section{Free induction decay and inversion recovery}
\subsection{}
The voltage signal collected by a coil placed near the sample is given by 
\begin{equation}
    emf = -\diff{}{t}\int_{sample} d^3r\overrightarrow{M}(\overrightarrow{r},t)\cdot\overrightarrow{B}^{receive}(\overrightarrow{r})
\end{equation}
In the calculation, assume the sample to be small enough and a coil placed on the $x$-axis.
Find the results in Fig. \ref{fig:figure4.1}. RF pulses are applied every 25 ms. 

\subsection{}
The results for different Inversion Time are shown in Fig. \ref{fig:figure4.2.1}.
The $M_\perp(T_I)$ vs $T_I$ is shown in Fig. \ref{fig:figure4.2.T1est}.

\section{Spin echo}
\subsection{}
A uniform distribution is not a good assumption for the isochromats frequencies at equilibrium.
For presentation purpose, the relaxation are slowed down by using $T_1$ = 20 ms and $T_2$ = 15 ms only in Problem 5.
The span of the frequencies are 0.05 ($\delta_\omega/\omega_0$).
The echo time $T_E$ = 15 ms.
The result is shown in Fig. \ref{fig:figure5.1}.
\subsection{}
Rejection method was used to generate random numbers following Lorentzian distribution. Same span was used, and $\Delta$ = 0.1 for the Lorentzian distribution. 
Find the result in Fig. \ref{fig:figure5.2}.
\subsection{}
Figure \ref{fig:figure5.1} shows considerable side packet in the free precession.
This is not the natural case when the material is approaching equilibrium under external magnetic field.
Figure \ref{fig:figure5.2} shows the correct response with a good estimation of isochromats frequencies.
\subsection{}
Find the result in Fig. \ref{fig:figure5.4}.
\pagebreak

\image{figure2.4}{pdf}{Forced precession in the rotating frame.}
\image{figure3.1}{pdf}{Forced precession in the laboratory frame.}
\image{figure3.2}{pdf}{Free precession with relaxation in the laboratory frame.}
\image{figure3.3}{pdf}{Free precession with relaxation in the laboratory frame.}
\image{figure3.4}{pdf}{Free precession with relaxation in the laboratory frame, ahead of resonance frequency.}
\image{figure3.5}{pdf}{Free precession with relaxation in the laboratory frame, behind resonance frequency.}
\image{figure4.1}{pdf}{Free induction decay sequece.}
\image{figure4.2.1}{pdf}{IR signal. $T_I = 0.5 T_1\ln2$}
\image{figure4.2.2}{pdf}{IR signal. $T_I = T_1\ln2$}
\image{figure4.2.3}{pdf}{IR signal. $T_I = 1.5 T_1\ln2$}
\image{figure4.2.T1est}{pdf}{$M_\perp(T_I)$ vs $T_I$ for $T_1$ estimation}
\image{figure5.1}{pdf}{Spin echo, assuming unifor distribution.}
\image{figure5.2}{pdf}{Spin echo, assuming unifor distribution.}
\image{figure5.4}{pdf}{Hahn spin echo, assuming unifor distribution.}

\end{document}